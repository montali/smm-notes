%\begin{example} (In 2 dimensions)\\
    % Let $\vec{y} \in \set{R}^2$ and $L=\text{span}\{\vec{u}\}$ one-dimensional subspace of $\set{R}^2$ generated by $\vec{u} \in \set{R}^2$. We can write $\vec{y}$ as
    % $$ \vec{y} = \hat{\vec{y}} + \vec{z} $$
    % and notice that $\vec{u} \parallel \hat{\vec{y}}$.
    
    % The projection $\hat{\vec{y}}$ is the closest vector lying in the subspace $L$ to $\vec{y}$, where distance is measured by $\norm{\vec{y} - \hat{\vec{y}}}$. It follows that $\vec{z} = \vec{y} - \hat{\vec{y}}$ is orthogonal to $L$ and, in particular, to $\vec{u}$, i.e. $\langle\vec{z}, \vec{u}\rangle = 0$.
    
   % Since $\hat{\vec{y}}$ is an element of $L$ it is a multiple of $\vec{u}$
    %$$ \hat{\vec{y}} = \alpha\vec{u} $$
    
    We can find $\alpha$ in the following way
    $$ \langle\vec{x} - \zz, \vec{b}\rangle = 0 \implies \langle\vec{z} - \alpha\vec{x}, \vec{b}\rangle = 0 \implies \langle\vec{x}, \vec{b}\rangle - \alpha\langle\vec{b}, \vec{b}\rangle = 0 $$
    $$ \alpha = \frac{\langle\vec{x}, \vec{b}\rangle}{\langle\vec{b}, \vec{b}\rangle} = \frac{\transp{\vec{x}}\vec{b}}{\norm{\vec{b}}^2} $$
%\end{example}